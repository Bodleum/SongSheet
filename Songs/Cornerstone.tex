\begin{song}{Cornerstone}
    \verse
    {My hope is built on nothing less}
    {Than Jesus blood and righteousness}
    {I dare not trust the sweetest frame}
    {But wholly trust in Jesus name}
    \end
    \chorus
    {Christ alone, Cornerstone,}
    {Weak made strong, in the Saviour's love}
    {Through the storm, He is Lord,}
    {Lord of all.}
    \end
    \verse
    {When Darkness seems to hide His face}
    {I rest on His unchanging grace}
    {In every high and stormy gale}
    {My anchor holds within the veil}
    \end
    \verse
    {When He shall come with trumpet sound,}
    {O, may I then in Him be found.}
    {Dressed in His righteousness alone,}
    {Faultless stand before the throne.}
    \end
\end{song}

% ---

